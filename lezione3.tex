\section*{Lezione 3}
\addcontentsline{toc}{section}{Lezione 3}

\subsection*{Codici pesati}
\addcontentsline{toc}{subsection}{Codici pesati}
Dato un messaggio da spedire nel canale, si associa ad esso un insieme di valori.
Ad esempio, dato un messaggio $m = m_1 \; m_2 \; ... \; m_n$, codifichiamo ogni $m_i$ nel seguente modo:
\begin{center}
	26 lettere A-Z\\
	1 spazio \textvisiblespace\\
	10 cifre 0-9\\
	$\downarrow$\\
	37 simboli
\end{center}

Successivamente si associa ad ogni messaggio un insieme di valori di \textit{peso}:

\begin{center}
	 msg: $m_1, \; m_2, \; ... \; m_n, \; c$\\
	 pes: $n+1, \; n, ... \; 2, \; 1 $
\end{center}

Facciamo un esempio, mettiamo di voler spedire la parola \texttt{ciao}, utilizzando la codifica mostrata in precedenza.

\begin{table}[h]
	\centering
	\begin{tabular}{llllll}
		& \texttt{C} & \texttt{I} & \texttt{A} & \texttt{O}  & check \\
		\hline
		msg & 2 & 8 & 0 & 14 & check \\
		pes & 5 & 4 & 3 & 2  & 1    
	\end{tabular}
\end{table}

Il mittente pesa ogni carattere molitplicando la codifica per il peso:

\begin{equation*}
	2 \cdot 5 + 8 \cdot 4 + 0 \cdot 3 + 14 \cdot 2 + check \cdot 1
\end{equation*}

Succcessivamente si pone questa espressione uguale a 0 \textit{mod} 37 (questo è il numero totale di caratteri).

\begin{equation*}
2 \cdot 5 + 8 \cdot 4 + 0 \cdot 3 + 14 \cdot 2 + check \cdot 1 = 0 \; mod \; 37
\end{equation*}
\begin{equation*}
70 + check = 0 \; mod \; 36
\end{equation*}
\begin{equation*}
check = 4
\end{equation*}

Successivamente spedisce il messaggio con il vettore dei pesi.\\
Il ricevitore si calcola il peso (usando il valore di \textit{check} spedito dal mittente) e controlla che il peso sia uguale a 0 \textit{mod} 37.

Da notare che 37 è primo, quindi ci va bene perchè lavoriamo in un campo.\\
Ad esempio i conti in modulo 6 hanno i divisori di zero (3 $\cdot$ 2 = 0 mod 6).
Se non è primo piuttosto aumentiamo il numero di caratteri (piuttosto non li usiamo, ma ci semplifica tutto avere un numero primo).

I codici pesati sono utili per individuare errori come ad esempio lo scambio di caratteri, l'inserimento di caratteri da parte del canale ecc...

\medskip

Ad esempio, se all'interno di un messaggio al posto di \texttt{ab} viene inviato \texttt{ba}, allora la differenza fra le somme pesate è (fissato \textit{k+1} come peso di \texttt{a}, \textit{k} come peso di \texttt{b}).

\begin{equation*}
a(k+1)+bk \rightarrow b(k+1)+ak
\end{equation*}
\begin{equation*}
a(k+1)+bk - b(k+1)+ak = 
\end{equation*}
\begin{equation*}
= b(k+1)+ak - a(k+1) -bk
\end{equation*}
\begin{equation*}
= bk +b+ak-ak-a-bk
\end{equation*}
\begin{equation*}
= b-a
\end{equation*}

Quindi il messaggio viene accettato solo se la differenza è uguale a 0, quindi quando $b-a = 0 \; mod \; 37$, ma per essere 0 allora i due caratteri devono essere uguali.
Se due simboli uguali vengono scambiati quindi, il messaggio viene accettato lo stesso.

Vediamo un algoritmo per calcolare i pesi:

\begin{algorithm}
	\begin{algorithmic}[1]
		\Procedure{CalcolaCheckDigit}{}
		\State $\textit{sum} \gets 0$
		\State $\textit{ssum} \gets 0$
		\While{\textit{not} EOF}
		\State read $\textit{symbol}$
		\State $sum \gets sum + \textit{symbol} \; \text{mod} \; 37$
		\State $ssum \gets ssum + \textit{sum} \; \text{mod} \; 37$
		\EndWhile
		\State $temp \gets ssum + sum \; \text{mod} \; 37$
		\State $c \gets 37 - temp \; \text{mod} \; 37$
		\State return $c$
		\EndProcedure
	\end{algorithmic}
\end{algorithm}

Una simulazione dell algoritmo con la parola \texttt{ciao} è la seguente:

\begin{table}[h]
	\centering
	\begin{tabular}{l|l|l}
		msg & sum & ssum   \\
		\hline
		\texttt{C} = 2  & 2   & 2      \\
		\texttt{I} = 8  & 10  & 12     \\
		\texttt{A} = 0  & 10  & 22     \\
		\texttt{O} = 14 & 24  & 46 $\equiv$ 9
	\end{tabular}
\end{table}
\vspace{-5mm}
\begin{equation*}
temp \leftarrow 9 + 24 \equiv 33\\
\end{equation*}
\begin{equation*}
c \gets 37 - 33 = 4 \equiv \texttt{E}
\end{equation*}

\medskip

Quindi il messaggio spedito sarà \texttt{ciaoe}.

\newpage

Il ricevente ricalcolerà la tabella usando il messaggio appena arrivato:

\begin{table}[H]
	\centering
	\begin{tabular}{l|l|l}
		msg & sum & ssum   \\
		\hline
		\texttt{C} = 2   & 2   & 2      \\
		\texttt{I} = 8   & 10  & 12     \\
		\texttt{A} = 0   & 10  & 22     \\
		\texttt{O} = 14   & 24  & 46 $\equiv$ 9 \\
		\texttt{E} = 4  & 28  & 37 $\equiv$ \textbf{0}
	\end{tabular}
\end{table}

La somma delle somme finale è 0, quindi il messaggio è ok. Se non fosse 0, allora il messaggio è stato modificato, ma non posso sapere dove.

Astraiamo il procedimento appena utilizzato:

\begin{table}[H]
	\centering
	\begin{tabular}{l|l|l}
		msg & sum & ssum   \\
		\hline
		\texttt{w}  & $w$   & 2      \\
		\texttt{x}  & $w+x$  & $2w+$x     \\
		\texttt{y}  & $w+x+y$ & $3w+2x+y$    \\
		\texttt{z}  & $w+x+y+z$  & $4w+3x+2y+z$ \\
		\texttt{c}  & $w+x+y+z+c$  & $5w+4x+3y+2z+c \equiv 0 \; mod \; 37$ 
	\end{tabular}
\end{table}

Si noti come la somma delle somme assegna un peso decrescente a tutti i caratteri: 5 a \textit{w}, 4 a \textit{x} e così via...

Il codice ISBN è un esempio di codice pesato su base 11; proviamo a verificare il seguente codice: 1-58488-508-4

\begin{table}[H]
	\centering
	\begin{tabular}{l|l|l}
		msg & sum & ssum   \\
		\hline
		\texttt{1}  & 1   & 1      \\
		\texttt{5}  & 6  & 7     \\
		\texttt{8}  & $14 \equiv 3$ & 10   \\
		\texttt{4}  & $7$  & $17 \equiv 6$ \\
		\texttt{8}  & $15 \equiv 4$  & 10 \\
		\texttt{8}  & $12 \equiv 1$  & $11 \equiv 0$ \\
		\texttt{5}  & $6$  & 6 \\
		\texttt{0}  & $6$  & $12 \equiv 1$ \\
		\texttt{8}  & $14 \equiv 3$  & 4 \\
		\hline
		\texttt{4}  & $7$  & $11 \; mod \; 11 \equiv \textbf{0}$ \\
	\end{tabular}
\end{table}

Quindi il codice è corretto. Nel caso l'ultima cifra fosse un 10 si inserisce una \texttt{x}.

\newpage 
\subsection*{Es. 2.4 - 1, pag.24}
Caso: rumore bianco\\
$p=0.001$ (probabilità errore in ogni posizione)\\
$n=100$ (dimensione pacchetto)\\

Quanto è la probabilità che ci siano 0 errori?

\begin{equation*}
p(\text{0 errori}) = (1-p)^n
\end{equation*}
\begin{equation*}
= (1-\frac{1}{1000})^{100}
\end{equation*}
\begin{equation*}
= e^{100 \cdot ln(1 - \frac{1}{1000})}
\end{equation*}
\begin{equation*}
\approx e^{100 \cdot ln(-\frac{1}{1000})}
\end{equation*}
\begin{equation*}
= e^{-\frac{1}{10}}
\end{equation*}

\subsection*{Es. 2.4 - 2, pag.25}
Caso: rumore bianco\\
$p=0.001$ (probabilità errore in ogni posizione)\\
$n=100$ (dimensione pacchetto)\\

Quanto è la probabilità di non riuscire ad individuare un errore?

Per rispondere a questa domanda, possiamo rispondere a: quanto è la probabilità che il controllo di errore singolo sbagli?

Ricordando l'equazione (23):

\begin{equation*}
p(\text{controllo parità fallisce}) = \frac{1 + (1-2p)^n}{2}-(1-p)^n
\end{equation*}
\begin{equation*}
= \frac{1 + (1-\frac{2}{1000})^{100}}{2}-(1-\frac{1}{1000})^{100}
\end{equation*}
\begin{equation*}
\approx 0.0045
\end{equation*}

\subsection*{Codici a correzione d'errore}
I codici a correzione d'errore, a differenza del controllo di parità, permette al ricevente di rilevare dove è avvenuto un errore ed eventualmente di correggerlo.
Questo ovviamente richiede di aggiungere delle cifre di controllo ulteriori.

\newpage
\subsubsection*{Codici rettangolari - 1 Errore}
Il messaggio inserito nel canale viene rappresentato come un rettangolo (logicamente parlando, non fisicamente).
Rappresentiamo i bit di messaggio in questo modo:
\medskip
\begin{equation*}
\begin{matrix}
\textit{o} & \textit{o} & \textit{o} & ... & \textit{o} & x\\
\textit{o} & \textit{o} & \textit{o} & ... & \textit{o} & x\\
\vdots & \vdots & \vdots & \ddots & \vdots & \vdots\\
\textit{o} & \textit{o} & \textit{o} & ... & \textit{o} & x\\
x & x & x & ... & x & 
\end{matrix}
\end{equation*}
Nell'esempio il carattere \textit{o} rappresenta un bit di messaggio, \textit{x} rappresenta un carattere di controllo.
Ogni cifra di controllo sulle righe codifica la riga corrispondente, quelle sulle colonne codificano la colonna.

In questo modo, se un bit di messaggio viene modificato, posso individuare subito l'errore in quanto verranno sbagliati i bit di controllo sulla relativa riga e colonna.
Ad esempio, se un bit in posizione $i, j$ viene alterato, allora i due controlli di parità sulla riga $i$ e sulla colonna $j$ falliranno, ma il ricevente intuisce subito quale bit ha subito modifiche.\\

Per capire quanto il metodo è efficiente, andiamo a calcolarne la ridondanza:

\begin{equation}
R = \frac{m \cdot n}{(m-1)(n-1)}
\end{equation}
\begin{equation*}
= 1 + \frac{1}{m-1} + \frac{1}{n-1} + \frac{1}{(m-1)(n-1)}
\end{equation*}

Quindi l'eccesso di ridondanza equivale a $\frac{1}{m-1} + \frac{1}{n-1} + \frac{1}{(m-1)(n-1)}$.\\
Quanto è l'eccesso di ridondanza più piccola possibile in questa classe di codici (codici rettangolari)?
L'idea è quella di minimizzare la funzione
\begin{equation*}
\frac{1}{m-1} + \frac{1}{n-1} + \frac{1}{(m-1)(n-1)}
\end{equation*}
Si dovrebbero quindi calcolare le derivate parziali rispetto a \textit{n} e a \textit{m} e vedere quando si annullano, ma è facilmente intuibile che questa quantità è la minore possibile quando $m=n$.

Per cui i più efficienti sono i codici \textbf{quadrati}, ovvero il sottoinsieme di codici rettangolari in cui il numero di righe è uguale al numero di colonne.

La loro ridondanza vale
\begin{equation*}
R = \frac{n^2}{(n-1)^2} = 1 + \frac{2}{n-1} + \frac{1}{(n-1)^2}
\end{equation*}

I peggiori invece, sono i codici con una singola riga (i bit di controllo sono di più dei bit di messaggio!)













