\section*{Lezione 14 - Parte 2}
\addcontentsline{toc}{section}{Lezione 14 - Parte 2}

\subsection*{Crittografia}
\addcontentsline{toc}{subsection}{Crittografia}

Che cos'è la crittografia? Lo studio delle tecniche che permettono di elaborare, memorizzare e trasmettere dati in presenza di agenti ostili (una persona o dispositivo che cerca di leggere dati salvati, disturbare le computazioni ecc..).

Il nome deriva dalla composizione di due parole greche (\textit{kryptòs}: nascosta e \textit{graphia}: scrittura).
Da non confondere con la steganografia (nascondere messaggi dove solo in pochi sanno dove andare a cercare).
Il messaggio in crittografia è pubblico a tutti, ciò che è nascosto è il significato del messaggio (o il contenuto).

Prima del 1976 esistevano solo crittosistemi simmetrici, da quell'anno nasce la crittografia a chiave pubblica.


La crittografia ha due obiettivi:
\begin{itemize}
	\item Studiare e implementare \textit{crittosistemi}
	\item Analizzare crittosistemi con l'obiettivo di trovare vulnerabilità (\textit{crittoanalisi})
\end{itemize}

Tipi di cifrari:
\begin{itemize}
	\item Simmetrici: chiavi per cifrare e decifrare uguali (DES, 3DES, AES)
	\item Stream
	\item Chiave pubblica: RSA, ElGamal,...
\end{itemize}

Primitive crittografiche:
\begin{itemize}
	\item Funzioni hash
	\item Generatori di numeri pseudo-casuali
\end{itemize}

Protocolli:
\begin{itemize}
	\item Firme digitali
	\item Scambio di chiavi
	\item Autenticazioni
	\item Scambio di segreti
\end{itemize}

\subsubsection*{Principio di Kerkhoffs}
\addcontentsline{toc}{subsubsection}{Principio di Kerkhoffs}

Il principio di Kerkhoffs dice che è inutile tener nascosto l'algoritmo di cifratura e quello di decifratura. La parte da tener nascosta è la più piccola possibile, la chiave.

\begin{center}
	\textit{La segretezza non dovrebbe risiedere nelle funzioni E() e D(), ma piuttosto in una piccola quantità di informazione chiamata chiave}
\end{center}

Gli algoritmi sono visibili a tutti quindi (se lo rompi pubblichi e diventi famoso).

Un \textbf{crittosistema} è una quintupla formata dai seguenti elementi:

\begin{itemize}
	\item PT: spazio di tutti i testi in chiaro (plaintext)
	\item CT: spazio di tutti i testi cifrati (ciphertext)
	\item K: spazio delle chiavi (keyspace)
	\item Funzioni di cifratura (E: PT $\times$ K $\rightarrow$ CT)
	\item Funzioni di decifratura (D: CT $\times$ K $\rightarrow$ PT)
\end{itemize}

Come notazione si dice che $E_k(x)$ dove $k$ è la chiave e $x$ è il testo da cifrare.

Lo spazio delle chiavi non dovrebbe essere troppo piccolo, altrimenti la chiave può essere trovata attraverso un attacco di forza bruta.