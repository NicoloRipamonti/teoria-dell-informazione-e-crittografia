\section*{Lezione 20}
\addcontentsline{toc}{section}{Lezione 20}

\subsection*{Esponenziazione modulare}
Riprendiamo con un esempio: il gruppo moltiplicativo sarà $\mathbb{Z}_5^* = 1,2,3,4$, che è contenuto in $\mathbb{Z}_5 = {0,1,2,3,4}$.

2 è un generatore di $\mathbb{Z}_5^*$, infatti:

\[
\mathbb{Z}_5^* = \{2^0, 2^1, 2^2, 2^3\} = \{1,2,4,3\}
\]

Quindi possiamo definire la seguente funzione di esponenziazione modulare:
\begin{equation*}
f(z) = 2^z \text{ mod } 5
\end{equation*}

Nessuno è riuscito a trovare un algoritmo in P per trovare $z$ a partire da $f(z)$.
Dato $x$, però, posso calcolare $g^x \text { mod } p$ in tempo polinomiale (rispetto all'input). Ha senso considerare l'esponente $x$ fra 0 e $p-2$ (minore di 0 va all'inverso, maggiori di $p-2$, riconsidero gli stessi elementi visto che c'è modulo).
La dimensione dell'input è il numero di bit che servono per rappresentare gli elementi di $\mathbb{Z}_p^*$, quindi circa $n=log_2p$.\\
Possiamo quindi scrivere:
\begin{equation*}
	x = \sum_{j=0}^{n-1}x_j2^j
\end{equation*}
Dove $(x_{n-1}, ..., x_1, x_0)$ è la rappresentazione binaria di $x$.

Uno pseudocodice potrebbe essere:

\medskip
	\begin{algorithmic}
	\State \texttt{result} = 1
	\While {$x > 0$}
	\State \texttt{result} = (\texttt{result} * $g$)\text { mod }$p$
	\State $x = x - 1$
	\EndWhile
	\State{\textbf{return} \texttt{result}}	
\end{algorithmic}

\medskip
Però questo algoritmo chiede un numero esponenziale di iterazioni in quanto $x \approx 2^n$.\\


Però vale che:
\begin{equation*}
	g^x = g^{\sum_{j=0}^{n-1}x_j2^j} = \prod_{j=0}^{n-1} g^{x_j2^j}= \prod_{j=0}^{n-1}(g^{2^j})^{x_j} \; \; \text{ mod } p
\end{equation*}
Osservando l'ultima parte della formula posso calcolarmi tutti i valori di $g^{2^j}$ per $j \in \{0,1,...n-1\}$ e moltiplico fra loro solo quello per cui vale $x_j = 1$. Questo può essere fatto in maniera polinomiale attraverso il metodo \textit{square and multiply}.

\medskip

	\begin{algorithmic}
	\State {\texttt{result} $=1$}
	\For{$j=n-1 \text { downto } 0$}
	\State {\texttt{result} = (\texttt{result} * \texttt{result) mod $p$}}
	\If{$x_j = 1$}
	\State {\texttt{result} = \texttt{result} $* g$ mod $p$}
	\EndIf
	\EndFor
	\State{\textbf{return} \texttt{result}}		
	\end{algorithmic}

\medskip

L'inverso dell'esponenziazione modulare è chiamato \textit{logaritmo discreto}.
Il problema è così definito:
\begin{itemize}
	\item Numero primo $p$
	\item Generatore di $\mathbb{Z}_p^*$ chiamato $g$
	\item $y \in \mathbb{Z}_p^*$
\end{itemize}
Si computa una $x \in \{0, 1, ..., p-2\}$ tale per cui $g^x \equiv y \text{ mod } p$.

La computazione del logaritmo discreto è un problema intrattabile in tempo polinomiale. Per avere un idea della difficoltà del problema, prova a calcolare a mano i logaritmi base 3 dentro a $\mathbb{Z}_{113}^*$.

Abbiamo quindi trovato una scatola dove nascondere un valore $x$ che non può più essere aperta; in altre parole abbiamo trovato una one-way function. Manca però una \textbf{trapdoor} che consenta a Bob di trovare una soluzione in modo facile.

Un altro esempio di funzione che sembra essere one-way è la moltiplicazione fra numeri naturali: dati due numeri primi $p$ e $q$, calcolare il prodotto $n = p \dot q$.
Il problema inverso è la fattorizzazione: dato un numero interno $n$ che è il prodotto di due numeri primi, trovarli (ogni numero ammette un unica rappresentazione in fattori primi).\\



Un esempio di fattorizzazione:\\



RSA-768=12301866845301177551304949583849627207728535695953347921973224521\\
517264005072636575187452021997864693899564749427740638459251925573263034\\
537315482685079170261221429134616704292143116022212404792747377940806653\\
51419597459856902143413 = \\
334780716989568987860441698482126908177047949837137685689124313889828837\\
93878002287614711652531743087737814467999489 × 3674604366679959042824463\\
37996279526322791581643430876426760322838157396665112792333734171433968\\
10270092798736308917


L'algoritmo "banale" è quello che prova a dividere $n$ dato in input per tutti gli interi da 2 a $\sqrt{n}$, ma che equivale a un tempo di:

\begin{equation*}
	O(\sqrt{n}) = O(\sqrt{2^m}) = O(2^{\frac{m}{2}})
\end{equation*}

che è esponenziale rispetto a $m$ (il numero di bit).

\subsection*{Algoritmo di Diffie-Hellman}
\addcontentsline{toc}{subsection}{Algoritmo di Diffie-Hellman}

Vogliono sfruttare queste "mancanze" matematiche in crittografia. Alice e Bob scelgono (in maniera pubblica) un numero primo $q$ e un generatore $g$ del gruppo ciclico $\mathbb{Z}_q^*$. Poi alice sceglie a caso un $x_A$, con $0<x_A<q-1$, e lo tiene segreto. Bob allo stesso modo sceglie un $0 < x_B < q-1$ e lo tiene segreto.

Ora:
\begin{enumerate}
	\item Alice calcola $g^{x_A}$ mod $q$ e lo manda a Bob. In altre parole nasconde $x_A$ dentro una "cassetta".
	\item Bob calcola $g^{x_B}$ mod $q$ e lo manda ad Alice.
	\item Alice calcola $(g^{x_B})^{x_A} = g^{x_A \cdot x_B} = k$
	\item Bob calcola $(g^{x_A})^{x_B} = g^{x_B \cdot x_A} = k$ 
	\item $k$ è il valore della chiave segreta
\end{enumerate}
Eve vede solamente $q, g, g^{x_A}, g^{x_B}$, ma per trovare $x_A$ o $x_B$ deve risolvere un logaritmo discreto! Anche calcolare $g^{x_A \cdot x_B}$ partendo da $g^{x_A}$ e $g^{x_B}$, ma anche questo sembra un problema intrattabile.

\subsubsection*{Esempio}

Alice e Bob si mettono d'accordo utilizzando $q=25307$ e $g=2$ (che è un generatore di $\mathbb{Z}_{25307}^*)$.
Alice sceglie $x_A = 3578$ e Bob $x_B = 19956$.
Alice computa $g^{x_A} = 2^{3578} = 6113$ mod 25307 e lo manda a Bob. Allo stesso modo Bob calcola $g^{x_B} = 2^{19956} = 7984$ mod 25307 e lo manda ad Alice.
Quando Bob riceve $g^{x_A} = 6113$ da Alice, egli calcola $k=(g^{x_A})^{x_B} = 6113^{19956} = 3694$ mod 25307, e come lui anche Alice calcola $(g^{x_B})^{x_A} = 7984^{3578} = 3694$ mod 25307, che è la stessa chiave $k$ calcolata da Bob.
Ora Alice e Bob possono usare $k$ come chiave segreta in un crittosistema simmetrico

\subsection*{Crittosistema El Gamal}
\addcontentsline{toc}{subsection}{Crittosistema El Gamal}

Ogni utente sceglie un numero primo $q$, e considera un generatore $g$ del gruppo ciclico $\mathbb{Z}_q^*$.
Come prima sceglie un numero $a$ tale che $0 < a < q-1$ e computa $g^a$ mod $q$. La chiave segreta sarà $a$, e verrà nascosta dentro $g^a$ mod $q$. La chiave pubblica è $k_p(q, g, g^a)$. Per ottenere la chiave segreta dalla chiave pubblica dovremmo computare $log_gg^a$ mod $q$.\\


Alice vuole mandare un messaggio $m$ a Bob (questa $m$ dev'essere un elemento di $\mathbb{Z}_q^*$ quindi dev'essere minore di $log_2q$ bit, altrimenti lo divide in blocchi).
Lei prende la chiave pubblica di Bob $k_{pB} = (q, g, g^a)$. Ora sceglie un valore $0 < \ell < q-1$ e calcola $\gamma = g^{\ell}$ mod $q$ e $\delta = m \cdot (g^a)^{\ell}$ mod $q$.
Poi manda a Bob il testo cifrato $c=(\gamma, \delta)$.
Il valore $\ell$ rende i testi cifrati non correlabili fra loro (anche se sono sempre uguali).
La decifratura funziona in questo modo:
\begin{equation*}
\delta = m \cdot (g^a)^{\ell} \text{ mod } q.
\end{equation*}
\begin{equation*}
m = g^{-a\ell} \cdot \delta
\end{equation*}
\begin{equation*}
	m = g^{-a\ell} \cdot \delta = \gamma^{-a} \cdot \delta = \gamma^{q-1-a} \cdot \delta
\end{equation*}

Eve, come prima, dovrebbe risolvere un logaritmo discreto per rompere il crittosistema.

Questi crittosistemi a chiave pubblica però come si può vedere sono molto più lenti di quelli simmetrici (centinaia di volte più lenti).
Quindi si creano dei crittosistemi ibridi che sfruttano le potenzialità di entrambi: si usa un crittosistema a chiave pubblica per decidere una chiave segreta (chiave di sessione), e poi si utilizza questa in un crittosistema simmetrico. In questo modo si utilizza il crittosistema lento solo per passarsi qualche centinaia di bit.

\subsection*{RSA}
\addcontentsline{toc}{subsection}{RSA}

Dati due numeri primi molto grandi $p$ e $q$ della stessa dimensione più o meno (almeno 1024 bit ciascuno), ma valori molto diversi.\\
Si pone $n=pq$ ($n$ chiamato \textit{modulo} di RSA).
Ora si crea la funzione toziente di Nepero $\phi(n) = \phi(pq) = \phi(p)\phi(q) = (p-1)(q-1)$.\\
$\phi(n)$ è il numero di interi compresi fra 1 e $n$ che sono \textit{coprimi} con $n$ ($\phi(n) = |\{x:1 \leq x \leq n \text{ and MCD}(x,n) = 1\}|$).\\
In altre parole $\phi(n)$ è il numero di elementi contenuti in $\mathbb{Z}_N^*$.\\


Si sceglie un $d$ a caso tale che $1 < d < \phi(n)$ e che MCD$(d, \phi(n)) = 1$, quindi $d$ è invertibile modulo $\phi(n)$ (quindi è invertibile in $\mathbb{Z}_{\phi(N)}$)

Usando l'algoritmo di Euclide esteso si computa $e$ tale per cui $e \cdot d \equiv 1 \text{ mod } \phi(n)$ (che vuol dire $e \equiv d^{-1} \text{ mod } \phi(n)$)

La chiave pubblica è data dalla coppia $(n, e)$, mentre la chiave segreta sarà $d$. Ovviamente i valori di $p, q, $ e $\phi(n)$ devono rimanere segreti.

\begin{itemize}
	\item \textbf{Cifratura}:\\
	$c=m^e$ mod $n$\\
	Dove $m$ ed $e$ fanno parte della chiave pubblica. Se la lunghezza del messaggio $m$ da cifrare ha una dimensione più grande di $n$ dev'essere diviso in blocchi.
	\item \textbf{Decifratura}:\\
	$c^d$ mod $n = m$\\
	Essa funziona in questo modo:
	\begin{equation*}
		c^d \text{ mod } n \equiv (m^e)^d \equiv m^{ed} \text { mod } n
	\end{equation*}
	Ora, siccome $ed\equiv 1$ mod $\phi(n)$, per definizione di congruenza esiste una $k$ intera tale per cui $ed = 1 + k\phi(n)$, da cui:
	\begin{equation*}
		c^d \text{ mod } n \equiv m^{1+k\phi(n)} \equiv m \cdot (m^{\phi(n)})^k \text { mod } n
	\end{equation*}
Per il piccolo teorema di Fermat sappiamo che $m^{\phi(n)} \equiv 1$ mod $n$, quindi:
\begin{equation*}
	c^d \text{ mod } n \equiv m \cdot (1)^k \equiv m \cdot 1 \equiv m \text{ mod } n
\end{equation*}
	
\end{itemize}
Si sfrutta quindi l'esponenziazione modulare che è una permutazione one-way (la trapdoor è la chiave segreta).

\subsubsection*{Esempio}
Bob sceglie $p=101$ e $q=113$, da cui $n_B=pq=11413$ e $\phi(n_B) = (p-1)(q-1) = 11200$.\\
Ora deve scelgliere una $d_b$ fra 2 e $\phi(n_B) - 1 =11199$ tale che l'MCD fra $d_B$ e $\phi(n_B)$ sia uguale a 1.\\
Mettiamo che scelga $d_B=6597$.\\
Usando l'algoritmo di Euclude esteso, Bob si calcola
\begin{equation*}
	e_B \equiv d_B^{-1} \text{ mod } \phi(n_B) = 3533 \text { mod } 11200
\end{equation*}
Quindi la chiave pubblica è data dalla coppia $(n_B, e_B) = (11413, 3533)$,\\
la chiave privata è $d_B=6597$.\\

Se Alice volesse mandare il messaggio $m=9726$ e Bob, allora si calcola $c=9726^{3533}$ mod 11414 = 5761 e manda $c$ a Bob.\\
Egli recupera $m$ calcolando:
\begin{equation*}
	c^{d_B} = 5761^{6597} \text{ mod } 11413 = 9726
\end{equation*}

Ovviamente nella realtà le dimensioni dei numeri sono di almeno 1024 bit.\\


Quanto è difficile per Eve conoscere $\phi(n)$? \'E facile se conosciamo la fattorizzazione di $n$:
\begin{equation*}
	\phi(n) = n \prod_{p|n}(1 - \frac1p)
\end{equation*}
\'E difficile se partiamo da $n$ senza conoscere la sua fattorizzazione: in altre parole rompere RSA è equivalente a fattorizzare $n$.

Se Eve riuscisse a scoprire $\phi(n)$ può fattorizzare $n$:
\begin{itemize}
	\item Sa che $n=p\cdot q$ e che $\phi(n) = (p-1)(q-1) = pq - (p+q)+1=n-(p+q)+1$
	\item Calcola $p+q=n-\phi(n) + 1$
	\item Ora che conosce la somma e il prodotto di $p$ e di $q$, li calcola risolvendo la seguente equazione quadratica $x^2 -(p+q) = 0$.
\end{itemize}

I valori di $p$ e $q$ non dovrebbero essere troppo vicini fra loro (altrimenti si avvicinano alla $\sqrt{n}$ e quindi provando a cercare intorno a quel valore troviamo uno dei due fattori e fattorizziamo $n$).
Vediamo un caso in cui $p$ e $q$ sono vicini:
\begin{itemize}
	\item Assumiamo che $p>q$
	\item Se i due sono vicini, allora $\frac{p-q}{2}$ è piccolo, e $\frac{p+q}{2}$ è leggermente più grande di $\sqrt{n}$.
	\item Dalla sequenze uguaglianza (che è valida in generale):
	\begin{equation*}
		\frac{(p+q)^2}{4} - n = \frac{(p-q)^2}{4}
	\end{equation*}
	Deduciamo che $\frac{(p+q)^2}{4}$ è un quadrato perfetto.
	\item Allora cerchiamo degli interi $x > \sqrt{n}$ tali per cui $x^2-n$ è un quadrato perfetto che chiamiamo $y^2$.
	\item Da $y^2 = x^2 -n$ otteniamo:
	\begin{equation*}
		n = x^2-y^2 =(x+y)(x-y)
	\end{equation*}
	E quindi abbiamo fattorizzato! $p=x+y$ e $q=x-y$
\end{itemize}


Dovremmo anche occuparci del valore di $\phi{n}$: assumiamo che il MCD è grande, allora
\begin{equation*}
	u = \text{mcm}(p-1,q-1) = \frac{(p-1)(q-1)}{\text{MCD}(p-1,q-1)} = \frac{\phi(n)}{\text{mcm(p-1, q-1)}}
\end{equation*}
Dalle proprietà dei campi se $d' \equiv e^{-1} \text{ mod } u $ si può utilizzare $d'$ per decifrare al posto di $d$. Ma siccome $u$ è piccolo, si può trovare a tentativi. Quindi è meglio se $p-1$ e $q-1$ abbiano divisori grandi.\\

Un altro errore è utilizzare lo stesso modulo $n$ per un gruppo di utenti.
Alice manda
\begin{equation*}
	c_1 = m^{e_1} \text { mod } n
\end{equation*}
\begin{equation*}
	c_2 = m^{e_2} \text { mod } n
\end{equation*}
Se MCD$(e_1, e_2) = 1$ (sono coprimi fra loro) allora Eve può usare il teorema di Euclide esteso e calcola $r$ e $s$ tali per cui $re_1 + se_2 = 1$.
Una volta calcolato $r$ e $s$ Eve può calcolare:
\begin{equation*}
	c_1^rc_2^s \equiv m^{re_1}m^{se_2} \equiv m^{re_1 + se_2} \equiv m \text { mod } n
\end{equation*}

Anche utilizzare esponenti piccolo per valori diversi del modulo può essere un problema: mettiamo che un utente vuole mandare $m$ a tre utenti $A,B,C$ usando tre moduli diversi:
\begin{equation*}
	c_A = m^3 \text{ mod } n_A
\end{equation*}
\begin{equation*}
	c_B = m^3 \text{ mod } n_B
\end{equation*}
\begin{equation*}
	c_C = m^3 \text{ mod } n_C
\end{equation*}

Se $n_A,n_B$ e $n_C$ sono coprimi allora Eve può usare il teorema dei Cinesi per calcolare un valore $m$ la cui radice cubica intera ritorna $m$.\\

Rompere RSA equivale a fattorizzare $n$.

\subsubsection*{Randomized RSA}

Un modo robusto di utilizzare RSA è la seguente: supponiamo di avere una sequenza di singoli bit da cifrare. Allora prendiamo la chiave pubblica di RSA $(n,e)$, allora se cifriamo 0 rimane 0, e così anche 1:
\begin{equation*}
	0^e \equiv 0 \text { mod } n
\end{equation*}
\begin{equation*}
	1^e \equiv 1 \text { mod } n
\end{equation*}
quindi il testo cifrato è uguale al testo in chiaro!
Si potrebbe risolvere questa cosa creando pacchetti di messaggi, ma magari non è possibile perchè devo mandarli immediatamente. Qualcuno ha però dimostrato che il bit meno significativo del testo in chiaro è un hard-core bit per l'esponenziazione modulare $m^e$ mod $n$ (quindi è difficile calcolare il bit meno significativo del testo in chiaro senza trapdoor, ovvero $d$ aka one-way function)
L'idea è quella di mettere il bit da cifrare come bit meno significativo, e gli altri bit vengono messi a random!
\subsubsection*{Esempio}
Alice vuole mandare a Bob un bit: prende la chiave pubblica di Bob $(n_B, e_B)$.
Sceglie poi un intero random $x < \frac{n_B}{2}$ (quindi $2x < n_B$).
Poi manda a Bob $y=(2x+b)^{e_B}$ mod $n_B$.\\
Bob quando riceve $y$ computa $y^{d_B}$ mod $n_B = 2x + b$ e prende il bit meno significativo del risultato.\\

Ovviamente questo è fattibile per cifrare qualche bit ogni tanto, e non per cifrare un bit alla volta un film.

